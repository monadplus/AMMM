
\documentclass[12pt, a4paper]{article}

\input{preamble}

\usepackage{fancyhdr}
\pagestyle{fancy}
\fancyhf{}
\rhead{TODO}
\lhead{TODO}
\rfoot{Page \thepage}

\title{%
  \vspace{-10ex}
  \Large{Algorithmic Methods for Mathematical Models (AMMM)} \\
  \large{Lab Session 3 - More on Mixed Integer Linear Programs}
}
\author{%
  Arnau Abella \\
  \large{Universitat Polit\`ecnica de Catalunya}
}
\date{\today}

\begin{document}
\maketitle

\vspace{10ex}
\begin{enumerate}[label=(\alph*)]
    \item Implement the P3 model in OPL and solve it using CPLEX with the following data file.

    The OPL model can be found at \code{computer.mod} file.

    The results of the execution are the following

    % \inputminted{modula2}{computer.mod}
    \begin{minted}{text}
Problem: test.dat
262414.256133333

Task - Computer
[1,0,0,]
[0,0,1,]
[0,1,0,]
[1,0,0,]

Thread - Core
[0,1,0,0,0,0,0,0,0,]
[1,0,0,0,0,0,0,0,0,]
[0,0,0,0,0,0,1,0,0,]
[0,0,0,0,0,0,0,1,0,]
[0,0,0,1,0,0,0,0,0,]
[0,0,0,0,0,1,0,0,0,]
[1,0,0,0,0,0,0,0,0,]
[1,0,0,0,0,0,0,0,0,]
CPU 1 loaded at 0.483938801
CPU 2 loaded at 0.410988458
CPU 3 loaded at 0.532826052
    \end{minted}

    \newpage

    \item Generate instances of increasing size with the instance generator script and use the \textit{P3} model to solve them.

    The OPL script can be found at \code{script.mod} file.

    I will only attach the results of the small data (\code{small.dat}) but the script is prepared to also run the medium data (\code{mid.dat}) and big data (\code{big.dat}).

    \begin{minted}{text}
Problem: small_0.dat
21134171.221917309

Task - Computer
[0,1,0,0,0,]
[1,0,0,0,0,]
[0,0,0,0,1,]
[0,0,1,0,0,]
[...]

Thread - Core
[0,0,0,1,0,0,0,0,0,0,]
[0,0,1,0,0,0,0,0,0,0,]
[0,0,0,1,0,0,0,0,0,0,]
[0,0,0,0,1,0,0,0,0,0,]
[0,0,1,0,0,0,0,0,0,0,]
[...]

CPU 1 loaded at 0.455212688
CPU 2 loaded at 0.320378826
CPU 3 loaded at 0.270715188
CPU 4 loaded at 0.684153368
CPU 5 loaded at 0.670319832
    \end{minted}

    \item Modify the \textit{P3} model to maximize the number of computers with all their cores empty (\textit{P3a}).

    The OPL model can be found at \code{computer2.mod} file.

    \newpage

    \item Compare both models \textit{(P3 and P3a)} in terms of number of variables, constraints and execution time for the generated instances.

    \begin{table}[H]
      \begin{center}
        \begin{tabular}{l | c r}
          & P3 & P3a \\
          \hline
          Constraint & 32 & 38 \\
          Variables & 85 & 88 \\
          CPU Time & 0.185 & 0.16 \\
        \end{tabular}
      \end{center}
      \caption{Comparison on \code{test.dat}}
    \end{table}

    \begin{table}[H]
      \begin{center}
        \begin{tabular}{l | c r}
          & P3 & P3a \\
          \hline
          Constraint &  215 & 225 \\
          Variables & 1101 & 1106 \\
          CPU Time & 5.27 & 3.90 \\
        \end{tabular}
      \end{center}
      \caption{Comparison on \code{small.dat}}
    \end{table}

\end{enumerate}


\end{document}
