
\documentclass[12pt, a4paper]{article}

\input{preamble}

\usepackage{fancyhdr}
\pagestyle{fancy}
\fancyhf{}
\rhead{Arnau Abella}
\lhead{AMMM}
\rfoot{Page \thepage}

\title{%
  \vspace{-10ex}
  \Large{Algorithmic Methods for Mathematical Models} \\
  \large{Lab Session 4 - Greedy and Local Search Heuristics}
}
\author{%
  Arnau Abella \\
  \large{Universitat Polit\`ecnica de Catalunya}
}
\date{\today}

\begin{document}
\maketitle

\vspace{5ex}
\begin{enumerate}[label=(\alph*)]
    %%%%%%%%% a
    \item[\textcolor{QuestionBlue}{\textbullet}] \textcolor{QuestionBlue}{Prepare a pseudocode for the Greedy algorithm. Specify the greedy function.}

    \begin{algorithm}[H]
      \SetAlgoLined
      \DontPrintSemicolon
      \SetKwInput{Input}{Input}
      \SetKwInput{Output}{Output}
      \Input{
        \begin{itemize}
            \item A set $T$ of tasks, each task $t$ requires $r_{t}$ resources
            \item A set $C$ of computers, each computer $c$ has a capacity of $r_{c}$ resources
        \end{itemize}
      }
      \Output{
        A set $w$ of assignments, each assignment $\angles{t, c}$ associates a task $t$ with a computer $c$ \textit{s.t.} Each task $t \in T$ appears exactly once
      }
      $w \leftarrow \emptyset$\;
      \ForAll{$t \in T$}{
        $c^{min} \leftarrow q(t, w)$\;
        \If{$q(c^{\min}) = \infty$}{
          \Return{$INFEASIBLE$}\;
        }\;
        $w \leftarrow w \cup \{ \angles{t, c^{min}} \}$\;
      }

      \Return{$w$}\;
      \caption{Greedy Algorithm}
    \end{algorithm}

    \begin{figure}[H]
      \begin{align*}
        &q(t,w) = \min \{ q(\angles{t, c}, w) \ | \ c \in C \}\\
        &q(\angles{t, c}, w) =
          \begin{cases}
            \infty& if r_{t} > r_{c} - \sum_{t' \in w_{c}}r_{t'}  \\
            \frac{r_{t} + \sum_{t' \in w_{c}}r_{t'}}{r_{c}} & otherwise
          \end{cases}
      \end{align*}
      % \caption{Cost function}
    \end{figure}

    where $w_{c} = \{t_{i}, \ldots, t_{j} \} \subseteq T$ are the tasks assigned to computer $c$ in the partial solution $w$.

    %%%%%%%%% b
    \item[\textcolor{QuestionBlue}{\textbullet}] \textcolor{QuestionBlue}{
      Prepare a pseudocode for the Local search algorithm. What neighborhoods and exploration strategies are implemented?
    }

    \begin{algorithm}[H]
      \SetAlgoLined
      \DontPrintSemicolon
      \SetKwInput{Input}{Input}
      \SetKwInput{Output}{Output}
      \Input{A set $w$ of assignments, each assignment $\angles{t, c}$ associates a task $t$ with a computer $c$
      }
      \Output{A set $w'$ of assignments \textit{s.t.} $q(w') \le q(w)$}
      $w' \leftarrow w$\;
      \While{execution time $\le$ time limit}{
        $sC \leftarrow$ sort $w'$ by computer's load (desc)\;
        moves $\leftarrow \emptyset$\;
        \For(\tcc*[f]{list traversal, order matters}){$c \in sC$}{
          \ForEach{$t \in w'_{c}$}{
            \For{$c' \in sC \text{ s.t. } c' \le c$}{
              \ForEach{$t' \in w'_{c'}$}{
                $rc = r_{c} - \sum_{t'' \in w_{c}}r_{t''}$\;
                $rc' = r_{c'} - \sum_{t'' \in w_{c'}}r_{t''}$\;
                \If{$r_{t'} - r_{t} \le rc \land r_{t} - r_{t'} \le rc'$}{
                  $rc_{new} = rc + r_{t} - r_{t'}$\;
                  $rc'_{new} = rc' + r_{t'} - r_{t}$\;
                  improvement $= \min(rc_{new}, rc'_{new})$\;
                  \If{improvement $> 0$}{
                    $moves \leftarrow moves \cup \{\angles{t,c,t',c',\text{ improvement}}\}$\;
                  }
                }
              }
            }
          }
        }
        \If{moves $= \emptyset$}{
          \Return{$w'$}
        }\;
        $\angles{t,c,t',c',\_} \leftarrow \argmax \{\angles{\_,\_,\_,\_,\text{impr}} \leftarrow \text{moves} \}$\;
      }
      $w' \leftarrow w' \cup (w'_{c} \setminus \{ t \} \cup \{ t' \})$\;
      $w' \leftarrow w' \cup (w'_{c'} \setminus \{ t' \} \cup \{ t \})$\;
      \Return{$w'$}\;
      \caption{Local Search: Task Swapping}
    \end{algorithm}

    The \textit{local search} uses a \textit{task swapping} algorithm and the strategy performed is \textit{best improvement}. The \textit{python} code implements the same algorithm but also includes a \textit{first improvement} strategy.

    %%%%%%%%% c
    \item[\textcolor{QuestionBlue}{\textbullet}] \textcolor{QuestionBlue}{
      Generate instances of increasing size. Store these instances as they will be solved in the coming lab sessions.
    }

    %%%%%%%%% d
    \item[\textcolor{QuestionBlue}{\textbullet}]{\color{QuestionBlue}
      Solve the instances previously generated using

        \begin{itemize}
            \item Random only
            \item Greedy function only
            \item Greedy + Local search
        \end{itemize}

      Plot the quality of the solutions and time to solve. Select the best combination.
    }

    The \textit{randomized} version can be discarded because of the quality of the solutions (figure \ref{fig:greedy}).
    The \textit{greedy with local search (best improvement)} can be discarded because of the execution time, see figure \ref{fig:greedy}, which grows so fast that I couldn't even manage to run the smallest instances without hitting the soft time limit of the program.
    Between the \textit{greedy without local search} algorithm and the \textit{greedy with local search (first improvement)} it is more disputed.
    The version with local search always returns better solutions but the execution time grows some order of magnitudes faster.

    So, from the collected data, I would recommend using the \textit{greedy with local search (first improvement)} algorithm for small instances where the execution time is feasible and the \textit{greedy without local search} algorithm for big instances.

    %%%%%%%%% e
    \item Solve the instances previously generated using the ILP from lab session 2.
    Configure CPLEX to stop after 30min or GAP $\le 1\%$.

    \item[\textcolor{QuestionBlue}{\textbullet}] \textcolor{QuestionBlue}{
      Solve the instances previously generated using the ILP from lab session 2.
      Configure CPLEX to stop after 30min or GAP $\le 1\%$.
    }

    %%%%%%%%% f
    \item[\textcolor{QuestionBlue}{\textbullet}] \textcolor{QuestionBlue}{Plot the best combination for the Greedy and the ILP of quality of the solutions and time to solve.}

    The \textit{ILP} version performs better in both speed and solution's quality, see figure \ref{fig:ilp}. Although, this comparison is not fair at all,
    the greedy algorithm is executing non-optimized \textit{python} code which is per se 10-100x orders of magnitude slower than \textit{C} while the ILP version is executing
    highly-optimized \textit{C} code.

    \begin{figure}[H]
      \includegraphics[width=\textwidth]{greedy_execution_time}
      \includegraphics[width=\textwidth]{greedy_solution_quality}
      \caption{Comparison between the \textit{greedy algorithms}}
      \label{fig:greedy}
    \end{figure}

    \begin{figure}[H]
      \includegraphics[width=\textwidth]{ilp_execution_time}
      \includegraphics[width=\textwidth]{ilp_solution_quality}
      \caption{Comparison between the \textit{Greedy} and the \textit{ILP} algorithm}
      \label{fig:ilp}
    \end{figure}

\end{enumerate}
\end{document}
